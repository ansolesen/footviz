\documentclass[a4paper,12pt]{article}

\usepackage[utf8]{inputenc}
\usepackage[T1]{fontenc}
\usepackage[UKenglish]{babel}
\newcommand{\Mod}[0]{\text{ mod }}
\usepackage{siunitx}
\usepackage{color}
\usepackage[]{float}
\usepackage{caption}
\usepackage{subcaption}
\usepackage{float}
\usepackage{fancyvrb}
\usepackage{amssymb}
\usepackage{amsmath}
\usepackage{listings}
\usepackage{comment} 
\usepackage[]{lipsum}
\usepackage{subfiles}

\usepackage{graphicx}
\DeclareGraphicsExtensions{.png}
\graphicspath{ {./Figures/} }

\definecolor{dkgreen}{rgb}{0,0.45,0}
\definecolor{gray}{rgb}{0.5,0.5,0.5}
\definecolor{mauve}{rgb}{0.30,0,0.30}

% Default settings for code listings
% Default settings for code listings
\lstset{literate=% 
{Ö}{{\"O}}1 
{Ä}{{\"A}}1 
{Ü}{{\"U}}1 
{ß}{{\ss}}{ 1\negmedspace\,} 
{ü}{{\"u}}1 
{ä}{{\"a}}1 
{ö}{{\"o}}1 
{ø}{{\o}}{1\negmedspace\,} 
{Ø}{{\O}}{1\negthinspace\,\,} 
{å}{{\aa}}{1\negthickspace\,} 
{Å}{{\AA}}{1\negthinspace\;} 
{æ}{{\ae}}{1\negthinspace\;} 
{Æ}{{\AE}}{1\,\,},
  frame=tb,
  language=Java,
  aboveskip=3mm,
  belowskip=3mm,
  showstringspaces=false,
  columns=flexible,
  basicstyle={\small\ttfamily},
  numbers=left,
  numberstyle=\footnotesize,
  keywordstyle=\color{dkgreen}\bfseries,
  commentstyle=\color{gray},
  stringstyle=\color{mauve},
  frame=single,
  breaklines=true,
  breakatwhitespace=false
  tabsize=1
}


\title{Visualization of Football Data\\\rule{10cm}{0.5mm}}
\author{Rasmus Bo Adeltoft\\Sebastian Seneca Haulund Hansen\\Steffen Berg Klenow\\Christian Bjørn Moeslund\\Andreas Staurby Olesen\\Henrik Sejer Pedersen
\\Supervisor: Marco Chiarandini\\\rule{5.5cm}{0.5mm}\\}
\date{\today}

\begin{document}
\maketitle
\newpage
\section{Abstract}

\section{Preface}

\newpage
\tableofcontents
\newpage
\section{Introduction}
Data is collected at a rapidly increasing rate in all fields and it becomes necessary to present data in different ways in order for humans to make sense of
it. One way to do this is through data visualization. Visualization can help
human's understanding of large data sets, as the data can be summarized very
effectively, and patterns can quickly be recognized by humans. When making
visualizations it is important to understand how the human cognitive system
works, such that visualizations can be designed to make it easier for humans to
understand the data. In order to do this, we will apply principles from the field
of visualization to present football data. We will use tools such as R to process
data and plot static visualizations, and use D3 to make interactive and dynamic
visualizations.

Specifically, we will do this both by making visualizations that can help explore
the questions that we present below, and by doing exploratory analysis such
that new patterns can be discovered. The specific questions that we will be
investigating are:
\begin{itemize}
\item How does a team evolve throughout a season in terms of goals, points,
etc.?
\item How does a team’s playing style (for example passes, possession and tackles)
change throughout a match?
\item How does a winning team differ from a losing team?
\end{itemize}
During the visualization process we will consider different visualization techniques
and choose a suitable one based on principles and analysis tools given
by Tamara Munzner in “Visualization Analysis and Design” to make sure that
the data is presented in an accurate and easily understandable manner. This
includes considerations regarding the human cognitive system.

\section{Theory} %approx 10 pages total

\subsection{Something} %approx 7 pages
\subsubsection{Design Process} % 0,5 pages Steffen
\subfile{DesignProces}
\subsubsection{Design} %what, why, how 1 pages Sebasitan

\subsubsection{Data Types and Data Sets} % 1 pages Sebastian

\subsubsection{Idioms} %which graphs fits which situations. Relevant for us, Channels and Marks, 2 pages. Christian

\subsubsection{Analysis and Complexity} % 1,5 pages 

\subsubsection{Facets and View Manipulation} % 1,5 pages. Steffen
\subfile{"Facets and View Manipulation"}

\subsection{Exploratory Analysis} % 3 pages. Rasmus

\subsection{Tools and Technologies} %approx 4 pages.
%introduction to R and D3
\subsubsection{R} %intro, why, how +(lib), pros/cons, 2 pages %Andreas

\subsubsection{D3.js} % 2 pages. Henrik

\section{Results}

\section{Discussion}

\section{Conclusion}

\section{Usability}

\section{Bibliography}
 \bibliographystyle{alpha}
 \bibliography{bib}
\section{Appendix}

\section{Process Evaluation}







\end{document}