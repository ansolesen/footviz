\documentclass[Report.tex]{subfiles}
\begin{document}
	D3, short for Data-Driven Documents, is an open-source JavaScript library that allows Document Object Model (DOM) manipulation based on data.\\
	It is designed to work on most "modern" internet browsers, \textit{which generally means everything except IE8 and older versions.} \cite[Browser / Platform Support]{D3W}\\
	
	Like R, the code written using D3 will provide high code reusability, as different visualizations will likely include similar elements. Once a visualization is made, it can be kept up to date simply by updating the data files, or not necessary at all if data is gathered through an API.\\ 
	D3 assists with binding arbitrary data to selected DOM elements.\\ This is done through creating a selection, on which the \texttt{.data()} function is called. This function joins the selection with the array provided as parameter, returning a new selection. On this selection the \texttt{.enter()} and \texttt{.exit()} selection functions can be called, \texttt{.enter()} returning a selection placeholder nodes for the data that had no DOM elements in the selection, and \texttt{.exit()}, the selection of existing DOM elements, which had no data points in the selection. DOM elements can then be added or removed based on these selections.\\
	Although D3 can be used to create HTML tables etc. the most common use of D3 for visualization purposes, is through Scalable Vector Graphics (SVG).
	SVG is based on XML, and can be modified / styled on the go through JavaScript for editing the SVG elements, and CSS for styling them. The CSS styling can be set and/or modified at runtime through JavaScript as well. Because of this, the programmer have the possibility to create visualizations that react to what the user is doing in interesting ways, or create dynamic visualizations.\\
	Another reason to use SVG over raster image formats, is that it will look as intended no matter how much it is scaled, where raster images like png or jpeg become pixelated when scaled.
	One of the core features in d3 is the Scales sub-library, which provides functionality that can transform all data given to fit inside the given range, while still maintaining the relative position / size between elements.
	This is especially useful when designing responsive web pages, where dimensions of the SVG element may differ.\\
	D3's SVG sub-library brings generators for several non-standard shapes in SVG. These generators can generate SVG paths which resemble the shapes. Some of these shapes are lines, arcs and chords. These shapes can be modified with options such as interpolation, which can be used to smoothen lines in a line chart, while they are still styleable through CSS.\\
	
	While this way of doing visualizations is preferred in a lot of cases due to reusability, and user interactivity, working with large datasets can give unwanted results. 
	If the user has to load more than a few megabytes of data in order to create the visualization, it is going to take some time, and if that data result in a large amount of SVG elements, a dynamic or interactive visualization could become unresponsive, which is undesired, as it will lower user experience.
\end{document}

