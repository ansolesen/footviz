\documentclass[Report.tex]{subfiles}
\begin{document}
As Tamara Munzner describes, \emph{data semantics} and \emph{data types} are important. In short, these are vital for humans’ ability to understand the data. The \emph{semantics} of the data is what the data represent in the real world. A \emph{table} with a row containing 5 different numeric values is useless unless the person working on it knows what these numbers represent. The \emph{type} of the data is equally important, as this is vital in understanding how to work with the given data. If it is a number, is it a quantity or an ID? This information is key, as adding two numeric values together makes sense for quantities, but would be useless for ID or code.

There are five basic data types: \emph{Items}, \emph{attributes}, \emph{links}, \emph{positions} and \emph{grids}. An \emph{item} is an individual entity, like a row in a table, since a single row in a table only describes one unique item in the given table. An \emph{attribute} is a measurable property, such as the base salary in different professions, a person’s height and so on. A \emph{link} is the relationship between items in networks, an example being the relationships between people in a city, such that a link describes the relationship between the citizens (Neighbours, family, acquaintances, friends…). A \emph{position} is spatial data, describing a location in two-dimensional or three-dimensional space. Here, a great example is the earth, which is divided into latitude and longitude coordinates, such that every point on earth has positional data, or a coordinate. Finally, a \emph{grid} describes the strategy for sampling continuous data in geometric and topological relationship between its cells.

The four basic data set types are \emph{tables}, \emph{networks}, \emph{fields} and \emph{geometry}. Other possibilities include \emph{clusters}, \emph{sets} and \emph{lists}. However, it is common for these basic types to be combined in real-world situations. Following will be a short description of each data set type. 
In a \emph{table}, each row represents an item of data, and each column is an attribute of the data set. Each individual cell contains a value of the pair of the item and the attribute.


\emph{Networks} and \emph{Trees} specify the relationship between two or more items. An item in a network is commonly referred to as a \emph{node}, and a \emph{link} is the relation between two nodes. 
A network with a hierarchical structure is called a \emph{Tree}. Unlike general networks, trees do not have cycles, as every child has only one node pointing to it, referred to as its parent node.


The \emph{field} data set type consists of cells. It is similar to tables, as it is build by keys and attributes. The main difference between fields and tables is that fields are used for continuos data. Each cell contains measurements from a continuous domain. In a way, there is an infinite amount of values to measure. Examples of this in the real world would be measuring the temperature. 
Finally, the \emph{geometry} data set type describes information about the shape of items. The items can range from one-dimensional lines to three-dimensional reconstructions of buildings and landmarks.  

\end{document}