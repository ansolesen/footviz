\documentclass[Report.tex]{subfiles}

\begin{document}

\begin{figure}
\center
\includegraphics[width=\textwidth, trim=0 700 250 0, clip]{"Succes Rates Season".pdf}
\caption{Comparison of a Teams success rates throughout the season, split in wins, draws and losses \url{http://footviz.copus.dk/radarchart2/}}
\label{Fig:CC}
\end{figure}

\paragraph{What-why-how\\}
Figure \ref{Fig:CC} shows the multi-view Radar Chart comparing a team's success rates. Concerning the marks and channels, we use position on a common scale in terms of the different axis of measure together with area, as the combined area of all the scores on each axis. The result of the match is categorised by color hue, in easy separable colours.  

The action of the idiom is to compare a team's success rates throughout the season in wins, losses and draws. It summarises the team's success throughout the season.
The target is to find similarities between the different matches in each of the match results, and maybe find some common shapes for the different teams.

This is done by an idiom faceted into three views. The views are linked by having the same data variables shown, but on different data. The view is manipulated by animation, looping over the different matches for the selected team. It is a smooth transition between the matches, to create context. The raw data is reduced through aggregation by calculating new attributes.

In R we load the data file, convert columns to be numerics, aggregate by team, calculate the success rates, transforming the data to a format suited for the radar chart, select the columns we are interested in, and write it to csv.

The D3 radar chart is based on \cite{Radar}, and modified to become animated. We create an update function which is called continuously with different data matches to be drawn. 

\paragraph{Findings\\}
In this idiom it is clear to see how much each team's success rates varies from match to match. Not only between the match results, but also in each category. It is interesting to see that some teams, even though they are loosing, have relative high success rates. The reason for this could be that they simply have been too passive, meaning their ball position is low, and only few, but successful actions have taken place.
\end{document}