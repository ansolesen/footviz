\documentclass[Report.tex]{subfiles}
\begin{document}
In modern day society, data is collected at a rapidly increasing rate in all fields and it has become a necessity to present data in different ways in order for humans to make sense of it. One way to do so is through data visualization. Visualization of data can help humans' understanding of large data sets, as the data can be summarized very effectively, and patterns can quickly be recognized. When making visualizations, there are several considerations the designer must make, including: What the semantics are of the data being visualized, what actions should be taken to reach the visualization, what is the target of the visualization and how the visualization is going to be created. It is also important to consider the human cognitive system when designing visualizations, such that the visualizations are designed to make it easier for humans to understand the data. Examples of this include limiting the amount of variables in the visualizations and highly contrasting colours. Given that visualizations are created to simplify large and complex data sets, sacrifices must be made to get the knowledge parsed to humans. Too complex visualizations will not accomplish the task of presenting the data in an easily understandable manner, as humans won’t be able to process all of the information. 
In order to design good visualizations, we will apply principles from the field of visualization to present football data. We will use tools such as the programming language R to process data and plot static visualizations, and use the JavaScript library D3.js to make interactive and dynamic visualizations. 
Specifically, we will do this both by making visualizations that can help explore the questions that we present below, and by doing exploratory analysis such that new patterns can be discovered. The specific questions that we will be investigating are: 
\begin{itemize}
\item How does a team evolve throughout a season in terms of goals, points,
etc.?
\item How does a team’s playing style change throughout a match?
\item How does a winning team differ from a losing team?
\end{itemize}
During the visualization process we will consider different visualization techniques and choose a suitable one based on principles and analysis tools given by Tamara Munzner in “Visualization Analysis and Design” to make sure that the data is presented in an accurate and easily understandable manner. This includes understanding the type of data that is to be visualized, considering how to filter the data to get the relevant data for the individual visualizations and how to create the visualization. We will discuss the results of the visualizations, specifically what we can and can’t learn from them, and reflect on how these visualizations could be used in other fields.
\end{document}