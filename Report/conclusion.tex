
\documentclass[11pt]{article}




\begin{document} 

\section*{Conclusion}

It turned out to be rather difficult to make any assured conclusion in regards to how a team evolves throughout a season as there turned out to be so many inconsistencies in most patterns which lead to a high margin of error. These inconsistencies seemed to be rather volatile, due to the fact that only 12 teams were compared, so a single team not fitting a pattern the hypothesis at proposal lead to a high divagation. The issue also applied whilst looking for differences between the top-tier and bottom-tier teams. Some tendencies were arguably found but were simply not supported well enough with evidence to actually conclude something concretely. However, some of the patterns which were found, seemed to indicated tendencies such that bottom-tier teams generally tended to shoot more, especially from outside of the box. Whilst looking at how a teams play-style changed throughout a match, we found a relatively clear indication that teams tend to shoot more frequently as the game progresses. The root of cause of the cause wasn't clear but one could suggest that either teams becomes more desperate to as the game progresses, or maybe the defense line in general terms cuts a bit more slack later into the game and make more pressure based errors. Speaking errors, it also seemed like teams were at higher risk of making mistaken passes at the beginning of each half, indicating that some factors must apply that causes players to preform slightly worse, which may be things as pressure or uncertainty of the opponent. Though these conclusions were supported by a substantial amount of quantitative records, some other tendencies were also found which were slightly less supported. The top-tier teams seemed to change their play-style in matches against bottom-tier teams, very much depended on what the outcome was. Especially factors such failed passes seemed to truly change depending on whom the opponent was, as it would often be higher if the opponent was worse on paper, in respect to playing against an on paper even opponent. Another tendency which was found, was that a losing team may start playing more towards the middle of the field whilst being behind in opposition to being a head. Although this theory is very poorly supported, it may very well lay grounds for a new hypothesis which is yet to be proved.
\end{document}