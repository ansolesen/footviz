
\documentclass[11pt]{article}




\begin{document} 



It turned out to be rather difficult to make any assured conclusion in regards to how a team evolves throughout a season as there turned out to be a lot of inconsistencies in most patterns we investigated, which lead to a high margin of error in the results. These inconsistencies seemed to be rather volatile, due to the fact that only 12 teams were compared, if a single team was not fitting the pattern our hypothesis had at proposal it lead to a high divagation. The issue also applied whilst looking for differences between the top-tier and bottom-tier teams. Some tendencies were arguably found but were simply not supported well enough with evidence to actually conclude something concretely. However, some of the patterns which were found, seemed to indicate a tendencies for bottom-tier team's to generally speaking shoot more, especially from outside of the box. Whilst looking at how a teams play-style changed throughout a match, we found a relatively clear indication that teams tend to shoot more frequently as the game progresses. The root of cause was not clear but one could suggest that either teams become more desperate as the game progresses, or maybe the defense line cuts a bit more slack later into the game and make more pressure based errors. It also seemed like teams were at higher risk of making unsuccessful passes at the beginning of each half, indicating that some factors must apply that causes players to perform slightly worse, which may be attributed to factors such as pressure or uncertainty of the opponent. Though these conclusions were supported by a substantial amount of quantitative records, some other tendencies were also found which were slightly less supported. The top-tier teams seemed to change their play-style in matches against bottom-tier teams but it depended heavily on what the outcome was. Especially factors such as failed passes seemed to truly change depending on whom the opponent was, as it would often be higher if the opponent was worse on paper, in respect to playing against an on paper even opponent. Another tendency which was found, was that a losing team may start playing more towards the middle of the field when being behind, as opposed to being ahead in the match. Although this theory is very poorly supported, it may very well lay grounds for a new hypothesis which is yet to be proven.
\end{document}