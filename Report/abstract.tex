\documentclass[Report.tex]{subfiles}

\begin{document} 
As data is being collected at a continuously increasing rate it has become a necessity to represent it in alternative manners for humans to make sense of it. One way of doing so is through visualization. Visualization is a powerful tool that augment human capabilities making it easier for us to interpreted data. However, there are many pitfalls and bad design decisions that may cause the visualizations to not perform as intended. In this project, we have been working with football data and tried to create new forms of visualization whilst also answering some common questions regarding how one can show the differences between a top-tier and bottom-tier team, whilst also looking at what their play-style is like, both in single matches and seasonal. To aid in creating visualization we have been using technologies such as \textit{R} and \textit{D3.js} for preprocessing and visualizing, respectively. Throughout the course of the project, we have also been using a variety of statistical methods to help finding patterns but also to prove some of the tendencies that were found along the way. We ended up with some visualizations, both new and common ones, which aided in finding patterns in the datasets. A pattern that we found was that more shots are actually taken towards the end of a game. The number of shots will already from the beginning of the game progressively increase all the way until the end. Some other tendencies the visualizations showed was that teams usually make more mistaken passes at the beginning of each half relative to the match, which may imply that some psychological factors affects the players and how they play. Another visualization also showed us that when a good team faces a worse one and loses, one can actually see a difference in how many failed passes they do relative to when they win. It also indicated that more play would develop around the middle of the field, which may indicate a rather frustrated play-style as the players are not using the field to its full extend. Even though these tendencies seemed clear in the last visualization, the data was only based off three matches which is not enough to confirm the hypothesis and thus, further investigation is needed. All said, it turned out to be rather difficult to compare anything concrete about the differences between top- and bottom-tier teams, as there were simply too much variation in the data.
\end{document}