%%This is a very basic article template.
%%There is just one section and two subsections.
\documentclass[Report.tex]{subfiles}

\begin{document}


In the world of data visualization, there are many different ways of presenting
data. In fact, there are so many ways, or idioms, that it is impossible to
comprehend the (dis)advantages of every single one. These idioms range from
simple static idioms, such as bar charts, scatter plots and line charts, to
more complex ones, such as static idioms linked together by interaction, such as
the comparison of success rates, described later. 

When selecting an idiom to represent a dataset, it is important to know the
strengths and weaknesses associated with that idiom. Different data sets
require different idioms, and in order to choose the right one, it is paramount
to know about marks and channels. 

\subsubsection{Marks and Channels}
Marks are basic graphic elements in a picture, while channels control the
appearance of the marks \cite[Chapter 5, p. 95-96]{Tamara}. In this respect,
marks can be anything from small points in a scatter plot, to lines in a line
chart, or even areas. Channels can be the position of the mark, the shape, tilt,
length or color of the mark, or anything else that alters the appearance of the
mark.
In a bar chart, the mark is typically a line with the channels being the length
of that line, as well as horizontal categorical information, for example the
different teams in our grouped bar charts. Color can also be used as a channel,
making it possible to encode more information in the chart, for example by
plotting different variables in a group, each variable having its own color. 

Not all channels can be used with all marks. If a two-dimensional mark is used
(area), it is not practical to use a shape-based visual channel, as it makes it
harder to compare the different marks with one another (comparing the area of a
triangle to the area of a rectangle, for instance). Therefore, if one runs out
of channels to encode information with, it may be wise to contemplate using
other kinds of marks, or even other visual channels, if the usage of one channel
means that other channels can't be used. 

The most effective channels should be matched with the most important
attributes, in order to increase perception of the visualization, and therefore
efficiency of the idiom. When the most important attributes are the easiest to
compare, it strengthens the visualization as a whole. 
The most effective magnitude channel (channels that describe quantity) is
aligned spatial position, with unaligned spatial position channels being second
best \cite[Chapter 5, p. 101]{Tamara}. After these comes length, angle and area,
length being more effective than area to convey quantitative data. 
The most effective identity channels (channels that describe categories) are
spatial regions, followed by color hue.

\subsubsection{Graphs}
\begin{itemize}
  \item \textbf{Scatter Plots}
  \\Scatter plots are effective when visualizing few (typically two) variables
  of a large dataset\cite[Chapter 5, p. 54]{Iliinsky}. Visual channels typically used in
  scatter plots include horizontal and vertical spatial position, color hue, area, as well as shape. However,
  if area is used on the marks, one should not use shape as an identity channel,
  but instead color hue. 
  Scatter plots can be used to quickly gain an overview of the data set, since
  it is possible to plot large quantities of observations.
  In our `Comparison viz` and our `Field Events`, a scatter plot is utilized to
  show the x and y coordinates of different events, and color hue is in the used
  in the former, to show if the match was won, tied or lost.
  
  \item \textbf{Line charts}
  \\Line charts are good for displaying continuous data or trends, and they can
  be much cleaner to look at, than bar charts. They usually have independent
  data on the x-axis, such as time or matches played, and dependent data on the
  y-axis, such as points in the league.
  
  \item \textbf{Bar charts}
  Bar charts are widely used for comparing variables, either across, or within
  categories. Later it is described how we utilize grouped bar charts to compare
  variables across teams. Groups are better than stacked bars, since stacked
  bars don't share the same baseline on the top bars.
  
  \item \textbf{Radar Chart}
  Radar charts are useful for comparing independent variables, such as the
  different success rates between teams. They can be used to show up to seven
  variables simultaneously and efficiently, quickly creating an overview of the
  differences in observations. For comparing dependent variables, such as a
  distribution, a histogram or a pie chart should be considered instead, as well
  as other idioms.

\end{itemize}


\end{document}
