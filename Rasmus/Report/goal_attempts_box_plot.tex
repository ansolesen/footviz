%Name in report: "Goal attempts box plot"
\documentclass[Report.tex]{subfiles}

\begin{document}

\begin{figure}
\center
\includegraphics[width=0.8\textwidth]{"goal_attempts_box_plot_with_interval".pdf}
\caption{Box plot of goal attempts in the first 10 minutes, the next 20 minutes and so on. The distance is the distance from the goal on the x-axis on the football field}
\label{Fig:goal_attempts_box_plot}
\end{figure}

To easily see how the goal attempts change throughout a match, the goal attempts of an entire season has been aggregated and then
split into intervals such that it is possible to see the distance from the goal a goal attempt is made during the first 10 minutes, the next
10 minutes and so on. This information has been visualized using a box plot (Figure \ref{Fig:goal_attempts_box_plot}). As described in the theory section, a box plot quickly allows a user to see different things about a data set, and therefore functions as a quick way to propose hypotheses about the data.

\end{document}